\chapter{Realizacja projektu}
\label{cha:realizacja}
Projekt została podzielony na kilka etapów w których realizacja każdego jest zależna od powodzenia poprzedniego. Najważniejszymi etapami okazały się: teoretyczne przygotowanie do zagadnienia wraz z doborem topologi sieci, implementacja algorytmu w środowisku MATLAB wraz z algorytmem uczenia sieci oraz ostateczna jego realizacja na układzie fpga. 
\section{Dobór topologii sieci}
Wybór odpowiedniej topologii sieci wymagał znajomości teoretycznej zagadnień związanych z algorytmami opartymi na sieciach neuronowych oraz wykorzystania własnego doświadczenia. Ze względu na niewystarczającą wiedzę w tym zakresie na początku projektu zdecydowano się na poszukiwaniu informacji o zagadnieniach zbliżonych. Okazało się, iż bardzo zbliżony projekt został już zrealizowany i opisany w <wstawić bib>. W artykule przedstawiono dokładnie etapy realizacji projektu którego celem było rozpoznawanie twarzy. Zawierał on również dokładny opis topologi sieci oraz sposób przygotowania danych do przetwarzania. 
Informacje te okazały się bardzo pomocne aby rozpocząć doświadczalny dobór topologi sieci, która pozwoli na poprawne przetwarzanie danych wejściowych. 
\section{Implementacja sieci neuronowej w MATLAB}

\section{Realizacja algorytmu w środowisku ISE}
Prawidłowo zrealizowany poprzedni etap projektu pozwolił na znaczne uproszczenie tej fazy ponieważ, posiadaliśmy dane na temat dobranej sieci oraz znaliśmy wszystkie wagi wejściowe każdego z neuronów co pozwoliło na pominięcie implementacji algorytmu uczenia sieci.


