\chapter{Podsumowanie}
\label{cha:podsumowanie}

\section{Wnioski}
Udało się zrealizować wszystkie założenia projektu. Uzyskany efekt jest bardzo zbliżony do rezultatów uzyskiwanych z algorytmów opartych na sztywnych przedziałach wartości RGB dla których powinna być wykrywana skóra. Zauważalną zaletą algorytmu opartego na sieciach neuronowych jest bezproblemowe wykrywanie skóry o różnych karnacjach. Aby uzyskać ten efekt należy użyć różnorodnych danych uczących, które zawierają próbki różnych odcieni skóry. Nie mniej jednak warto mieć na uwadze, iż udało się udowodnić jedynie poprawność działania algorytmów opartych na sieciach neuronowych. Jakość obrazu oraz dokładność rozpoznawania może zostać poprawiona przy użyciu filtrów medianowych oraz nie wyklucza się, iż zmiana topologii na bardziej złożoną może spowodować lepsze rezultaty. Zmiany te powinny zostać wprowadzone w pierwszym etapie dalszych prac nad projektem. Z powodu nie wystarczającej ilości czasu kolejne prace nad udoskonaleniem algorytmu nie zostały zrealizowane. 